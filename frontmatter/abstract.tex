\begin{abstract}
Tutorials are vital for helping people perform
complex software-based tasks in domains such as programming,
data science, system administration, and computational research.
However, it is tedious to create detailed step-by-step tutorials
for tasks that span multiple interrelated Graphical User Interface (GUI) and 
command-line  applications. Once these tutorials are created, it is 
also hard for tutorial creators to get fine-grained feedback about how
learners are actually stepping through their tutorials and which parts
lead to the most struggle.

We address these challenges by creating two prototypes:
\begin{itemize}
\item \textbf{Torta} - an end-to-end system that automatically generates
\rev{step-by-step GUI and command-line app tutorials} by demonstration, 
provides an editor to trim, organize, and add validation criteria to 
these tutorials, and provides a web-based viewer that can validate 
step-level progress and automatically run certain steps.
\item \textbf{Porta} - a system that automatically tracks how users navigate 
through a tutorial webpage and what actions they take on their 
computer. Porta tracks running shell commands, invoking compilers, and 
logging into remote servers and surfaces this trace data in the form of 
profiling visualizations. The visualization augments the tutorial with 
heatmaps of activity hotspots and markers that expand to show event details, 
error messages, and embedded screencast videos of user actions.
\end{itemize}

The core technical insight that underpins both these systems is that Operating-system-wide activity tracing makes it possible to easily generate new tutorials and profile existing tutorials.
\end{abstract}

An exploratory study on 10 computer science teaching assistants (TAs) 
and 6 students found that they all preferred the experience and results of 
using Torta to record and consume tutorials respectively.

A user study of 3 tutorial creators and 12 students who followed their
tutorials found that Porta enabled both the tutorial creators and the 
students to provide more specific, targeted, and actionable feedback about 
how to improve these tutorials.

These systems together open up possibilities of easing the use of creating quality step-by-step tutorials and user testing existing instructional material in a more systemic and scalable manner.