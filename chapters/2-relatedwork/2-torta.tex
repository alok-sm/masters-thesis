\section{Related work for Torta}

In addition to operating system monitoring, Torta builds upon ideas 
in three prior lines of work: mixed-media tutorials, generating tutorials 
by demonstration

\subsection{Mixed-Media Tutorials that Combine Text and Video}

The design of the Torta tutorial format was directly inspired by prior
work on \emph{mixed-media tutorials} that combine static (text+image)
and video modalities. These tutorials are formatted as a series of steps
on a webpage with a mix of textual exposition and embedded mini-videos
at each step. Chi et al.~\cite{Chi2012} performed a comparative study of
static, video, and mixed-media tutorials for image manipulation tasks
and discovered that users found mixed-media tutorials the easiest to
follow and the least error-prone. From this study they proposed four
design guidelines for mixed-media tutorials, which Torta follows:
1)~\emph{scannable steps}: Torta segments the creator's screencast
recording into steps based on active GUI windows, executed commands, and
text file edits, which facilitates scanning, 2)~\emph{small but legible
videos}: Torta crops each video segment to include only the active GUI
window, which emphasizes the most relevant portions at each step,
3)~\emph{visualize mouse movement}, and 4)~\emph{give control to the
user}: Torta tutorials are hierarchical, so more advanced users can hide
specific steps while novices can expand to see more details.

Inspired by this format, the Video Digests~\cite{Pavel2014} and
VideoDoc~\cite{Krosnick2014} systems semi-automatically generate
mixed-media tutorials from existing videos that contain time-aligned
transcripts. Although these were both designed for reformatting
lecture and talk videos, they could in principle be repurposed for
software tutorial videos as well.

In contrast to these projects, which aim to reformat existing video
tutorials, Torta allows users to make mixed-media tutorials from scratch
by recording a demonstration of their actions. Since Torta traces
OS-level metadata during the user demonstration, it can automatically
provide more specific details about software tutorial steps than text
transcripts or crowd workers' annotations can.


\subsection{Generating Tutorials by User Demonstration}

Torta's approach of generating tutorials via user demonstration of
actions was inspired by prior systems that operated in a similar way within
specific software applications.

Image editing applications have been the most popular substrates for
such demonstration-based tutorial generators. Grabler et al.\ built a
system that lets users generate photo editing tutorials by augmenting
the GIMP image editor to record application and UI state~\cite{Grabler2009}.
Chronicle~\cite{Grossman2010} similarly traces a detailed workflow history
within Paint.NET, another image editing application, to facilitate
tutorial creation and replay.
%
\rev{This was later turned into the Autodesk
Screencast~\cite{AutodeskScreencast} product, which records videos, metadata across supported Autodesk applications, app
switching and mouse/keyboard events, and provides a tutorial editor
similar to Torta's.}
%
Lafreniere et al.~\cite{Lafreniere2014} generated tutorials by
instrumenting Pixlr, an online image editing app, to collect usage logs
alongside a screencast video.
%
MixT~\cite{Chi2012} creates mixed-media tutorials by capturing
screencast videos alongside recorded application state and command logs
within Adobe Photoshop.

Other tools in this space also operate within specific apps or preset
collections of apps. For instance, ActionShot~\cite{Li2010} augments a
web browser to record a rich history consisting of in-browser actions
and form entries, which can be used to make tutorials for complex
web-based tasks. DocWizards~\cite{Bergman2005} instruments the Eclipse
IDE to let people generate tutorials for software development within
that IDE. InterTwine~\cite{Fourney2014} tracks user actions in both the
Firefox browser and the GIMP image editor to create tutorials that span
those two applications.

Torta differentiates itself by working across arbitrary sets of macOS
applications, whereas these prior tutorial generators operate either
wholly within a single application or on a small fixed set of apps.
Torta's design trades granularity for generality -- since it aims to be
generally applicable across all kinds of command-line and GUI apps, it
cannot do fine-grained tracing within specific apps like these related
projects can do.
