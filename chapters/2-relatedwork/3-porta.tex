\section{Related work for Porta}

Porta introduces the idea of tutorial profiling, which is inspired by
work in application usage profiling, information scent and improving web
tutorials apart from OS-wide activity tracing.

\subsection{Application-Specific Usage Profiling}

Porta extends the long lineage of systems that visualize \emph{usage
profiles} of how people use specific software applications.

Edit Wear and Read Wear~\cite{Hill1992} pioneered the use of graphical
marks on the scrollbar of a text editor to show where users were most
frequently editing or reading a document, respectively. Follow-up work
showed visualizations to help users re-find parts that they had
previously visited~\cite{Alexander2009}.
%
Patina~\cite{Matejka2013patina} is an application-independent
generalization of these ideas; it displays heatmaps over UI elements on
Windows applications to show which ones were frequently accessed.
%
Our Porta system displays a form of read wear on tutorial webpages in
the browser, which is powered by traces of how users
interacted with a multitude of \emph{other} applications on their computers.

%  - we're also a form of read-wear since we track how long you've been
%    lingering on each part of the webpage (subject to a 3-minute timeout
%    with no interactions or if the browser isn't the active window.
%    TODO: can we easily do this?) revisiting read wear

In a similar vein, LectureScape~\cite{KimUIST2014} collects usage
profiles of how people interact with educational videos. It displays a
histogram-like visualization overlaid on the scrubber timeline. Peaks in
this visualization represent where lots of viewers either paused or
navigated to, which could indicate potential points of high interest or
confusion. Porta also tracks video navigation but additionally
correlates that data with how users interact with other applications on
their computer.

ERICA~\cite{Deka2016} and ZIPT~\cite{Deka2017} capture usage profiles of
how users navigate through Android mobile apps. It displays flow
visualizations to help UX designers see which parts of the app their
users get stuck on. Porta is motivated by similar goals in how it
displays visualizations to show where users struggle when following
web-based software tutorials.

Code coverage and profiling tools~\cite{Graham1982,Srivastava1994} now exist for most
major programming languages. These tools show how many times each line of
code (or function call) ran and how much time it took. Theseus~\cite{Lieber2014} improves on
this idea by continually running user code and displaying
always-on profile visualizations in the margins of the code
editor. Bifr\"{o}st~\cite{McGrath2017} further extends code profiling to
mixed hardware-software systems. In contrast, Porta is a
tutorial profiler that tracks how a user ``runs" a software tutorial by
following it step-by-step and invoking various applications on
their computer.

%- code profilers exist for many languages such as gprof and in standard
%  libraries of languages such as python
%  (\url{https://docs.python.org/3.6/library/profile.html})
% Chrome DevTools: JavaScript CPU Profiling in Chrome 58
% https://developers.google.com/web/updates/2016/12/devtools-javascript-cpu-profile-migration

% 1982 gprof: A Call Graph Execution Profiler by Susan L. Graham, Peter
% B. Kessler, and Marshall K. Mckusick
%
% ATOM: A System for Building Customized Program Analysis Tools by
% Amitabh Srivastava and Alan Eustace

Porta innovates upon this class of prior work by introducing the novel
idea of building usage profiles for \emph{software tutorials} rather
than for specific pieces of software. It is designed to profile
tutorials whose instructions span multiple applications including web
browsers, IDEs, terminals, and commands executed on remote servers.
This makes it well-suited for technical tutorials in domains such as
software development, computational research, and system
administration.


\subsection{Multi-Application Information Scent}

The profiling sidebar that Porta displays over webpages provides
information scent~\cite{Pirolli2007} that helps tutorial creators hone in on
which parts their users struggled with. It was inspired by systems
that connect information scent across applications.

Codetrail~\cite{Goldman2009} and HyperSource \cite{Hartmann2011} augment
an IDE by connecting it to a web browser to link code and documentation.
Codetrail enables interactions such as automatically opening
documentation related to code that the user is currently editing.
HyperSource automatically associates code edits with webpages that the
user is currently viewing. Porta takes a complementary approach by
displaying multi-application scent in the browser and not being tied to
any specific IDE.

InterTwine~\cite{Fourney2014} connects a web browser and an image
editing application. One example form of information scent it provides
is to augment Google search result pages for image editing queries with
overlays of screenshots and metadata about how the user edited their own
images while reading those pages in the past. It requires the GIMP image
editor to be modified to add instrumentation code. In contrast, Porta
works across arbitrary command-line and GUI applications without needing
to modify their code, but its user activity tracing is not as deep as
what InterTwine provides for GIMP.


\subsection{Improving Web-Based Tutorials}

The goal of Porta is to help creators improve tutorials in response to
how users interact with their content. Porta takes an automated approach
by tracking activity across applications. In contrast, systems such as
LemonAid~\cite{Chilana2012} and TaggedComments~\cite{Bunt2014} opt for a
crowd-based approach by letting users directly annotate parts of
documentation and tutorial webpages that appear unclear to them. In the
future, we can add support for annotations to complement Porta's
automated tracking.

FollowUs~\cite{Lafreniere2013} implements a comprehensive solution by
embedding an instrumented version of a web-based image editing app
directly into tutorial webpages. This setup lets users follow
tutorials and post their own variant demonstrations directly on the
webpage
for future users to follow. Porta is motivated by similar goals but does
not require such specialized instrumentation. It works on existing
unmodified tutorial webpages and requires only installing a macOS tracer
app that monitors a variety of other apps without modifying them.
