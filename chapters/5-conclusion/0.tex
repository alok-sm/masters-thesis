\chapter{Conclusion}

% \input{chapters/5-conclusion/1-torta}
% \input{chapters/5-conclusion/2-porta}

To address the challenge of creating step-by-step tutorial with the same 
ease of creating a screencast, we created Torta, an end-to-end system 
for recording, editing, and consuming \rev{mixed-media tutorials that 
span multiple GUI and command-line applications. The core technical 
insight that underpins Torta is that the operating system already keeps 
track of vital filesystem and process-level metadata necessary for 
segmenting tutorial steps}. Thus, combining operating-system-wide activity 
tracing with screencast video recording makes it possible to quickly create 
complex GUI and command-line app tutorials by demonstration. \rev{Torta's
application-agnostic design} makes it well-suited for creating
tutorials in domains such as software development, data science, system
administration, and computational research.

To address the challenges of providing effective feedback on
the contents of software tutorials, we created Porta, a system
that automatically builds \emph{tutorial profiles} by tracking user activity
within a tutorial webpage and across multiple applications.
Porta surfaces these profiles as interactive visualizations that show
hotspots of user focus alongside details of logged application events
and embedded segments of recorded screencast videos. We found via a user
study of 3 tutorial creators and 12 learners that Porta helped both sets
of users reflect more concretely about what parts of the tutorials
caused the most trouble and what could potentially be improved. 

% Porta
% opens up possibilities for systematic user testing of technical
% documentation by providing fine-grained data to both test participants
% and tutorial creators.

% \rev{We hope that it will inspire the design of future tutorial systems
% that bridge the gap between video- and text-based formats.}

In conclusion, We believe that the systems described in this thesis make a very strong case toward the use of OS-wide monitoring for both creating new tutorials and profiling existing tutorials. 

Torta opens up the possibilities of future tutorial systems that could make 
it much easier to create step-by-step tutorials through demonstration, this 
allowing for the proliferation of quality instructional material

Porta on the other hand could enable low cost user testing of 
instructional material and could lead to improvements of existing tutorials 
and documentation online.