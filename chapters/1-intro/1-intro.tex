\chapter{Introduction}
It can be hard for experts in any field to write high-quality
documentation, instructional materials, and step-by-step tutorials. Instructional materials such as these are vital for helping people such as
software developers, data scientists, researchers, and system
administrators perform complex software-based tasks. 

One can either create a written tutorial by painstakingly enumerating 
all steps, describing shell commands, expected outputs, and side-effects,
and taking screenshots to illustrate GUI actions. Or one can demonstrate
all of the steps and record a screencast video, but it is tedious
to later fix bugs in videos. Neither approach is ideal for
creators. Also, neither is ideal for people trying to follow these
tutorials: manually-written tutorials may skip over critical details
~\cite{Lafreniere2013}, and videos can be difficult to browse and
navigate~\cite{Krosnick2014,Pavel2014}.

Additionally, once these tutorials are created, their creators cannot
easily get a sense of how people navigate through them or what parts 
they frequently get stuck on. They also cannot easily predict where 
new users might struggle~\cite{Lafreniere2013}. Despite 
the creator's best efforts, it is all too easy to omit certain tutorial
steps, gloss over subtle details, or provide incomplete explanations.
This is an instance of the \emph{expert blind spot} effect~\cite{Nathan2001}
where experts have trouble relating to what novices might know and not
know.

Thus, learners inevitably struggle because a) Quality tutorials are hard to come by because of the effort to create them. b) Tutorials that do exist have errors or bugs that the creators have no way to diagnose.

These limitations served as our research questions:
\begin{enumerate}
    \item Can we streamline the process for tutorial creation so that creating tutorial should be as easy as recording a screencast video, but tutorials should offer advantages of text-based formats like easy skimming and copy-paste.
    \item Can we provide effective feedback to tutorial creators about how learners are actually stepping through their tutorials and which parts lead to the most struggle?
\end{enumerate}

The core technical insight that underpins our solutions to these problems is
that \rev{the operating system already keeps track of vital filesystem and
process-level metadata necessary for creating and evaluating tutorials.}

The contributions of this paper are:

\begin{itemize}\itemsep0pt

\item [1a] A new approach to \rev{generating mixed-media tutorials by combining
screencast video recording with application-agnostic
operating-system-wide activity tracing}.

\item [1b] Torta (\textbf{T}ransparent \textbf{O}perating-system 
\textbf{R}ecording for \textbf{T}utorial \textbf{A}cquisition), a 
prototype system that instantiates this approach for macOS. Torta 
consists of an operating-system-wide activity recorder, a
tutorial editor, and a tutorial viewer that can validate step-by-step
progress and even run certain steps.

\item [2a] The novel idea of \emph{tutorial profiling} using operating-system-wide activity tracing.

\item [2b] Porta (\textbf{P}rofiling \textbf{O}perating-system 
\textbf{R}ecordings for \textbf{T}utorial \textbf{A}ssessment), a 
prototype that instantiates this idea for macOS. Porta consists of an
OS-wide activity recorder, a webpage navigation tracker, and interactive
profiling visualizations.

\end{itemize}

% \begin{enumerate}
%     \item \textbf{Torta} (\textbf{T}ransparent \textbf{O}perating-system \textbf{R}ecording for \textbf{T}utorial \textbf{A}cquisition), a system to automatically generated step-by-step mixed media tutorials by recording a screencast video along with OS-level events such as filesystem modifications and process invocations on the creator's machine

%     \item \textbf{Porta} (\textbf{P}rofiling \textbf{O}perating-system \textbf{R}ecordings for \textbf{T}utorial \textbf{A}ssessment), a system to provides feedback to tutorial creators about how users navigated through their tutorials and what application errors they encountered by recording a screencast video along with OS-level events such as filesystem modifications and process invocations on the learner's machine
% \end{enumerate}
