\begin{table*}[h]
\small % TODO: will this corrupt everything that comes afterward?
\centering
\begin{tabulary}{1\columnwidth}{@{}LL@{}|L@{}|L@{}}
\multicolumn{2}{@{}l|}{\textbf{Students}} & \textbf{Feedback on tutorial (before seeing Porta's output)} & \textbf{Feedback on tutorial given while using Porta} \\
\hline
P1 & Py   & ``easy to follow" & ``mention the variable in a `for' loop is a value and not an index" \\
P2 & Py   & ``tutorial is nice" & ``colons are used to start code blocks in Python" \\
P3 & Py   & ``I didn't know where to start writing the code" & ``boolean values start with capital letters in Python" \\
P4 & Py   & ``a summary of the tutorial would be good." & ``boolean values should be capitalized in Python" \\
\hline
P5 & Git  & ``seemed straightforward" & ``talk about how to exit the `less' program when showing `git status'" \\
P6 & Git  & ``need more Windows-specific advice" & ``explain what the `cat' command does here" \\
P7 & Git  & ``some parts did not clearly explain other tools used in the tutorial" & ``explain the git staging step better" \\
P8 & Git  & ``the language could be more novice friendly" & ``use a different formatting for text and commands" \\
\hline
P9  & Web & ``some screenshots were too small to read" & ``show large CSS changes that can be seen in screenshots" \\
P10 & Web & ``more details about Bootstrap" & ``add more comments in the starter code about what's going on" \\
P11 & Web & ``more explanation about what was supposed to happen" & ``no explanation of where event object for click handler comes from" \\
P12 & Web & ``more explanation about the lines they added" & ``make the file in which code is to be written more clean" \\

\\[0.5em] % newline

\multicolumn{2}{@{}l|}{\textbf{Instructors}} & \textbf{What they want to change in their tutorial (before using Porta)} & \textbf{What they want to change, mentioned while using Porta} \\

\hline

& & & ``talk about needing to start code blocks with colons \& indentation" \\
\multicolumn{2}{@{}l@{}|}{Python tutorial} & ``update this for Python 3" & ``split text into inline comments which the students actually read" \\
\multicolumn{2}{@{}l@{}|}{creator} & ``in the past, students had trouble with looping in this tutorial. look into how to do that part better." & ``didn't realize escape sequences threw people off; add more explanation about that" \\
& & & ``introduce basic Python syntax before talking about types" \\

\hline

& &  & ``use HEAD$\sim$1 instead of HEAD$\sim$ because it's more clear that you're going 1 commit back" \\
\multicolumn{2}{@{}l@{}|}{Git tutorial} & ``tutorial is probably too long" & ``don't use the intentional ``fil2" typo; none of the students got it" \\
\multicolumn{2}{@{}l@{}|}{creator} &``split some of the sections up" & ``make it more clear that students should use their own user names and email addresses in examples" \\
%& & & ``balance out copying with typing so students are forced to write some of their own commands" \\

\hline

& & & ``don't use placeholders in code; students copy them literally" \\
\multicolumn{2}{@{}l@{}|}{Web tutorial} & ``link to more external content" & ``show that capitalization matters when linking external files" \\ 
\multicolumn{2}{@{}l@{}|}{creator} & ``reformat some of the steps for better flow" & ``make more obvious CSS changes so students can actually see something happening" \\
& & & ``reverse the order of changes in the CSS of steps 7, 8, and 9" \\

\end{tabulary}

\caption{Examples of feedback given by the 12 students (top) and 3
instructors (bottom) on each tutorial, both before seeing Porta (left)
and while using Porta (right). Feedback while using Porta was usually
more concrete, specific, and precisely targeted to one particular
location within the tutorial.}

\vspace{-0.25em} % stent
\label{tab:study-feedback}
\end{table*}
